\subsection{Módulo de vídeo}

O módulo de vídeo é responsável pelo controle do modo de vídeo, dos \textit{backgrounds} e da renderização das \textit{sprites}.

Para gerenciar a renderização das \textit{sprites} foi desenvolvida uma classe chamada \textit{Texture}. Ela foi planejada de forma a não apenas copiar os dados da imagem para a região de memória apropriada, mas também permitir a animação das \textit{sprites} de uma textura e controlar os metadados das imagens renderizadas no jogo.

A seguir, será explicado, com o auxílo de trechos do código, o funcionamento dos principais elementos dessa classe.

Inicialmente, é necessário explicar os construtores dessa classe.

Logo abaixo, é possível visualizar o código do primeiro construtor. Ele recebe os ponteiros para a paleta de cores e para o vetor de \textit{tiles} utilizados pela imagem, os tamanhos das regiões de memória alocadas pra cada um desses ponteiros e a quantidade de \textit{bits} por \textit{pixel} que a imagem utiliza. Cada um desses atributos é guardado na própria classe, e, já nesse construtor, são chamados os métodos \textit{set\_sprite\_pal} e \textit{set\_sprite}, responsáveis por copiá-los para as regiões apropriadas. Por fim, ainda nesse construtor, são setados os índices do \textit{tile base} e da paleta de cores utilizada pela imagem. O cálculo desses índices será explicado logo adiante, nos tópicos dedicados aos métodos \textit{set\_sprite\_pal} e \textit{set\_sprite}.

\begin{lstlisting}[caption={Construtor da classe \textit{Texture}.}]
Texture::Texture(uint32_t num_sprites, const uint16_t *pallete, uint32_t pallete_len,
    const unsigned int *tiles, uint32_t tiles_len, enum bits_per_pixel bpp = _8BPP)
{
    this->pallete = pallete;
    this->pallete_len = pallete_len;
    this->pallete_id = 0;
    this->bpp = bpp;
    this->num_sprites = num_sprites;
    this->num_tiles = tiles_len / ((bpp == _4BPP) ? 32 : 64);
    this->tiles_per_sprite = num_tiles / num_sprites;
    this->tiles = tiles;
    this->tiles_len = tiles_len;

    memory_manager = MemoryManager::get_memory_manager();

    set_sprite_pal();
    set_sprite();
    oam_entry = memory_manager->alloc_oam_entry();

    metadata.tid = tile_base * ((bpp == _4BPP) ? 1 : 2);
    metadata.pb = pallete_id;
}
\end{lstlisting}
\vspace{\onelineskip}

O segundo construtor funciona de forma similar ao anterior, com a diferença de que em vez de receber todos os atributos da imagem, ele recebe apenas um ponteiro para outra textura. Esse construtor serve para quando se deseja renderizar réplicas de uma mesma textura. Utilizar ele permite que tais texturas compartilhem a paleta de cores e o vetor de \textit{tiles}, fazendo-se necessário alocar espaço apenas para os metadados, que são diferentes pra cada textura.

\begin{lstlisting}[caption={Construtor por cópia da classe \textit{Texture}.}]
Texture::Texture(const Texture *texture)
{
    this->num_sprites = texture->num_sprites;
    this->pallete = texture->pallete;
    this->pallete_len = texture->pallete_len;
    this->tiles = texture->tiles;
    this->tiles_len = texture->tiles_len;
    this->bpp = texture->bpp;
    this->pallete_id = texture->pallete_id;
    this->num_tiles = texture->num_tiles;
    this->tiles_per_sprite = texture->tiles_per_sprite;
    this->tile_base = texture->tile_base;

    memory_manager = MemoryManager::get_memory_manager();

    oam_entry = memory_manager->alloc_oam_entry();

    metadata.tid = tile_base * ((bpp == _4BPP) ? 1 : 2);
    metadata.pb = pallete_id;
}
\end{lstlisting}
\vspace{\onelineskip}

O método \textit{set\_sprite\_pal} é responsável por chamar o método \textit{alloc\_texture\_pal} da classe \textit{MemoryManager} passando como parâmetro o tamanho da paleta de cores a ser alocada. Como resultado, ele irá receber o endereço de memória aonde a paleta deverá ser guardada. Após fazer a cópia da paleta, é preciso calcular o índice da paleta escolhida na memória, já que esse é um dos metadados necessários para renderização da imagem a 4 \textit{bits} por \textit{pixel}. Como a região reservada para as paletas de cores no \textit{GBA} é divida em regiões de 32 \textit{bytes}, para realizar tal cálculo basta pegar a diferença entre o início da região escolhida para a cópia e o início da região reservada para as paletas e dividir por 32.
Como uma imagem renderizada a 8 \textit{bits} por \textit{pixel} ocupa toda a região reservada para as paletas, o cálculo do índice não é necessário para uma imagem que utilize 256 cores.

\begin{lstlisting}[caption={Função de alocação da paleta de cores de uma textura.}]
bool Texture::set_sprite_pal() {
    volatile uint8_t *teste = memory_manager->alloc_texture_palette(32);
    mem16cpy(teste, pallete, 32);

    this->pallete_id = (teste - (volatile uint8_t *)0x05000200) / 32;

    return true;
}
\end{lstlisting}
\vspace{\onelineskip}

O método \textit{set\_sprite} funciona de forma similar ao \textit{set\_sprite\_pal}, tendo como diferença relevante apenas o cálculo do índice do \textit{tile\_base} da imagem, que é feito subtraindo o início da região escolhida para a cópia e o início da região reservada para os \textit{tiles}. Essa diferença ocorre porque diferentemente da paleta de cores, cada unidade do vetor que representa a \textit{tile\_mem} no nosso código é, de fato, um \textit{tile}.

\begin{lstlisting}[caption={Função de alocação das \textit{sprites} de uma textura.}]
{c++}
bool Texture::set_sprite() {
    volatile struct tile *teste = memory_manager->alloc_texture(tiles_len);

    mem16cpy((volatile struct tile *)teste, tiles, tiles_len);
    tile_base = teste - memory_manager->base_texture_mem();

    return true;
}
\end{lstlisting}
\vspace{\onelineskip}

O método \textit{update\_metadata} apenas copia os metadados para a \textit{OAM} (\textit{Object Attributes Memory}).

\begin{lstlisting}[caption={Função de cópia dos metadados das texturas.}]
{c++}
void Texture::update_metadata() {
    mem16cpy(oam_entry, &metadata, sizeof(struct attr));
}
\end{lstlisting}
\vspace{\onelineskip}

Por fim, o método \textit{update} calcula qual a próxima sprite a ser renderizada e seta o primeiro \textit{tile} dela como \textit{tile\_id}. Esse processo é o que permite a animação das \textit{sprites} do jogo. O cálculo é feito somando o \textit{tile\_id} atual com a quantidade de \textit{tiles} por \textit{sprite} da textura que está sendo renderizada. Vale ressaltar que os índices dos \textit{tiles} são contabilizados sempre de 32 em 32 \textit{bytes}, mesmo que a imagem utilize 8 \textit{bits} por \textit{pixel} e por isso o número de \textit{tiles} por \textit{sprite} é multiplicado por 2 quando o número de \textit{bits} por \textit{pixel} da textura é 8.

Para a renderização dos \textit{backgrounds}, foi desenvolvida uma classe que recebe ponteiros para a paleta de cores,  para o vetor de \textit{tiles} e para o mapa de \textit{tiles} utilizados pelo \textit{background}, assim como os tamanhos das regiões alocadas pra cada um desses ponteiros. Assim que é instanciada, essa classe calcula qual o melhor \textit{charblock} e o melhor \textit{screenblock} para guardar os \textit{tiles} e o mapa de \textit{tiles}, respectivamente. Vale ressaltar que os \textit{charblocks} e \textit{screenblocks} compartilham a mesma região de memória e precisam de um espaço contíguo na memória do \textit{GBA} para que o \textit{background} seja renderizado corretamente. Por esse motivo não é recomendado apenas copiá-los para a memória do \textit{GBA} de forma sequencial, já que isso poderia causar um \textit{overlap} entre um \textit{charblock} e um \textit{screenblock}, e também poderia preencher a memória do \textit{GBA} de forma não-ótima, o que pode fazer com que não caibam todos os \textit{backgrounds} necessários para uma ou mais fases do jogo.