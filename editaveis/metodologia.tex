\chapter[Metodologia]{Metodologia}

A metodologia deste trabalho está descrita nas próximas seções e está dividida em Ferramentas de Desenvolvimento e Metodologia de Desenvolvimento.

\section{Ferramentas de desenvolvimento}

  Para a realização do porte do jogo para \textit{Game Boy Advance}, são necessários dois ambientes principais: um ambiente de desenvolvimento onde seja possível implementar o jogo e exportar o binário executável para o console e um ambiente para testar o executável gerado, sendo esse físico ou emulado.

  \subsection{Ambiente de desenvolvimento}

    O jogo será reescrito utilizando a linguagem C++, na versão 11, pois provê uma série de recursos e estruturas não presentes na linguagem C que facilitarão o desenvolvimento do jogo.

    O ambiente de desenvolvimento utilizado para a implementação do jogo consiste, basicamente, do \textit{kit} de desenvolvimento devkitARM.

    \subsubsection{\textit{devkitPro} e \textit{devkitARM}}

      O devkitPro\footnote{devkitPro, disponível em \url{https://devkitpro.org/}} é uma organização que provê conjuntos de ferramentas para desenvolvimento de jogos em diversos consoles da Nintendo, como Nintendo GBA, Nintendo Wii, Nintendo Switch, dentre outros.

      Dentre esses conjuntos de ferramentas encontra-se o devkitARM, \textit{toolchain} que contém o ambiente de desenvolvimento necessário para realizar a compilação do código escrito em C/C++ para a arquitetura de processadores ARM existente no GBA, citado no capítulo X.

    Para o ajuste das imagens do jogo para a resolução de tela do GBA, será utilizada a ferramenta de manipulação de imagens \textbf{GIMP}\footnote{GNU Image Manipulation Program, disponível em \url{https://www.gimp.org/}}, versão 2.8.

  \subsection{Ambiente de teste}

    \subsubsection{Emulador}

      Para a realização de testes com os executáveis gerados pelo devkitARM está sendo utilizado um emulador de \textit{Game Boy Advance}, chamado \textbf{VisualBoyAdvance-M}\footnote{VisualBoyAdvance-M, disponível em \url{https://github.com/visualboyadvance-m/visualboyadvance-m}}.

    \subsubsection{\textit{Console}}

      O \textit {console} que está sendo utilizado como ambiente de testes real é um \textbf{\textit{Nintendo DS}}\footnote{Nintendo DS, disponível em \url{https://www.nintendo.com/consumer/systems/selectds.jsp}}, que possui um \textit{slot} para cartuchos de \textit{Game Boy Advance}. Neste trabalho está sendo utilizado um cartucho especial onde é possível escrever arquivos executáveis diretamente nele.

      Para a escrita dos arquivos executáveis neste cartucho é utilizado o dispositivo \textbf{EZFlash II}\footnote{EZ Flash II, disponível em \url{http://www.ezfadvance.com/cards/EZ-Flash_2.htm}}. Como essa é uma versão antiga do produto, é necessário instalar um cliente para \textit{upload} dos arquivos para o cartucho. Este cliente só possui compatibilidade com \textit{\textbf{Windows XP}}\footnote{Sistema operacional da Microsoft, disponível em \url{https://support.microsoft.com/pt-br/help/14223/windows-xp-end-of-support}}, fazendo com que seja necessário instalar uma máquina virtual com o sistema operacional.

\section{Metodologia de desenvolvimento}

  \subsection{Desenvolvimento da \textit{Engine}}

    Para contribuir com uma arquitetura mais manutenível, foi optado por desacoplar a \textit{engine} do jogo em si. A \textit{engine} ficará responsável por implementar os módulos genéricos do jogo, enquanto que o jogo em si conterá as funcionalidades mais específicas.

    A \textit{engine} conterá uma classe que irá representar um objeto do jogo (do inglês, \textit{game object}). Esta classe ficará responsável por conter o comportamento genérico de um objeto dentro do jogo (podendo este ser um personagem, uma plataforma, um item coletável, etc.). Ele será representado da seguinte maneira:

    \vspace{\onelineskip}

    \begin{figure}[H]
      \centering \includegraphics[keepaspectratio=true,scale=0.6]{figuras/game-object.eps}
      \caption[Modelagem inicial da classe \textit{GameObject}]
        {Modelagem inicial da classe \textit{GameObject}. Fonte: \textit{Autores}.}
      \label{game-object}
    \end{figure}

    Onde, fora os metódos acessores e modificadores dos atributos da posição do objeto (\textit{x} e \textit{y}), têm-se o método \textbf{\textit{update()}}, que é puramente virtual e trata de qualquer atualização do objeto a cada frame e o método \textbf{\textit{draw()}}, também puramente virtual, que é responsável por renderizar o objeto na posição (\textit{x}, \textit{y}) a cada frame.

    É importante frisar que os métodos \textbf{\textit{update()}} e \textbf{\textit{draw()}} devem ser puramente virtuais, pois isso garante que qualquer classe que venha a extender de \textit{GameObject} seja obrigada a implementer suas próprias rotinas específicas de atualização e renderização.

    Além da classe \textit{GameObject}, os principais componentes da \textit{engine} a serem implementados são: módulo de vídeo, módulo de áudio, módulo de física e módulo de input.

    \subsubsection{Módulo de vídeo}

      O módulo de vídeo será responsável por renderizar qualquer tipo de imagem, animação e texto existente. Ele conterá as seguintes funcionalidades:

      \begin{itemize}
        \item Renderizar uma imagem de fundo;
        \item Renderizar uma \textit{sprite};
        \item Renderizar uma animação (como uma série de \textit{sprites});
        \item Movimentar horizontalmente uma imagem de fundo (\textit{horizontal scroll});
        \item Renderizar textos com uma determinada cor;
        \item Remover uma imagem, \textit{sprite}, texto ou animação que esteja renderizado da tela;
        \item Atualizar a renderização a cada frame, levando em consideração as posições \textit{x} e \textit{y} do \textit{sprite}, animação ou texto.
      \end{itemize}

      Deve-se lembrar que as imagens e textos que serão renderizados já estarão ajustados para a resolução e formato de cores corretos do GBA.

    \subsubsection{Módulo de áudio}

      O módulo de áudio ficará responsável por executar, no momento correto, qualquer música de fundo e efeito sonoro do jogo. Ele conterá as seguintes funcionalidades:

      \begin{itemize}
        \item Iniciar a execução de um efeito sonoro;
        \item Pausar a execução de um efeito sonoro;
        \item Parar a execução de um efeito sonoro;
        \item Iniciar a execução de uma música de fundo;
        \item Pausar a execução de uma música de fundo;
        \item Parar a execução de uma música de fundo.
      \end{itemize}

    \subsubsection{Módulo de física}

      O módulo de física terá como principal responsabilidade a detecção de colisões entre objetos do jogo. Ele conterá as seguintes funcionalidades:

      \begin{itemize}
        \item Simular, opcionalmente, a ação da gravidade em objetos do jogo;
        \item Detectar, opcionalmente, colisões entre objetos do jogo.
      \end{itemize}

    \subsubsection{Módulo de \textit{input}}

      O módulo de \textit{input} é responsável por receber qualquer pressionamento de qualquer um dos 10 botões e teclas do GBA.

  \subsection{Desenvolvimento do jogo}

    A implementação do jogo vai seguir o seguinte modelo de classes:

    \subsubsection{Objetos do jogo}

      O personagem principal, os itens coletáveis, plataformas, portais (para acesso e saída das fases) e elementos de HUD (\textit{heads-up display}) serão representadas como \textit{game objects}.

      Abaixo se encontra a modelagem inicial dos objetos do jogo.

      \begin{figure}[H]
        \centering \includegraphics[keepaspectratio=true,scale=0.6]{figuras/class-diagram-1.eps}
        \caption[Modelagem inicial dos objetos do jogo]
          {Modelagem inicial dos objetos do jogo. Fonte: \textit{Autores}.}
        \label{game-object-children}
      \end{figure}

    \subsubsection{Níveis}

      A classe \textit{Level} contém a generalização de um nível no jogo. No jogo, os menus (principal e de seleção de fases), cutscenes, fases e telas de vitória e derrota são representados como níveis do jogo.

      Abaixo se encontra a modelagem inicial dessas classes.

      \begin{figure}[H]
        \centering \includegraphics[keepaspectratio=true,scale=0.6]{figuras/class-diagram-2.eps}
        \caption[Modelagem inicial dos níveis do jogo]
          {Modelagem inicial dos níveis do jogo. Fonte: \textit{Autores}.}
        \label{game-object-levels}
      \end{figure}

    \subsubsection{Ajuste de recursos do jogo}

      Os recursos como imagens, áudios e arquivos de fonte do jogo precisarão ser editados para que possam ser carregados em memória e utilizados no GBA. No caso das imagens, por exemplo, serão modificadas características como dimensões e quantidade de cores a fim de diminuir seu tamanho para que possam ser convertidas para um formato utilizável no GBA.

\section{Cronograma de desenvolvimento}

  Abaixo se encontra o cronograma de desenvolvimento das tarefas planejadas para a conclusão do trabalho.

  \begin{figure}[H]
    \centering \includegraphics[keepaspectratio=true,scale=0.8]{figuras/cronograma.eps}
    \caption[Cronograma de desenvolvimento]
      {Cronograma de desenvolvimento. Fonte: \textit{Autores}.}
    \label{cronograma}
  \end{figure}