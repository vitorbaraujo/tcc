\subsection{Módulo de \textit{input}}

Os estados dos botões do GBA ficam salvos em um registrador. Cada um desses estados é representado por um \textit{bit} do valor guardado por esse registrador. Sempre que um botão é pressionado, o GBA automaticamente troca o valor guardado nesse registrador de tal forma que o \textit{bit} que representa o botão em questão passe a possuir valor 0. De forma similar, quando o botão é solto, o valor contido no \textit{bit} em questão é modificado para 1, seu valor padrão. Sendo assim, a checagem dos estados pode ser realizada facilmente utilizando \textit{bitmasks}.

Por exemplo, caso se deseje checar um botão representado pelo \textit{bit} 2 (com a contagem começando em 0), basta pegar o resultado do \textit{AND} binário entre o valor guardado no registrador e a potência de 2 que possui como expoente o \textit{bit} em questão (4, nesse exemplo). No Código \ref{lst:input1} é possível visualizar a definição das constantes que representam os botões. A função utilizada para checar o estado de cada um deles pode ser vista no código \ref{lst:lalala}.

\begin{lstlisting}[caption={Cabeçalho do módulo de \textit{input}.},label={lst:input1}]
#ifndef INPUT_H
#define INPUT_H

#include <stdbool.h>
#include "base_types.h"

#define BUTTON_A 1
#define BUTTON_B 2
#define BUTTON_SELECT 4
#define BUTTON_START 8
#define BUTTON_RIGHT 16
#define BUTTON_LEFT 32
#define BUTTON_UP 64
#define BUTTON_DOWN 128
#define BUTTON_R 256
#define BUTTON_L 512
#define N_BUTTON 10

int pressed_state[N_BUTTON];

void check_buttons_states();
bool pressed(int button);

#endif
\end{lstlisting}

\begin{lstlisting}[caption={Código-fonte do módulo de \textit{input}.},label={lst:lalala}]
int pressed_state[N_BUTTON];
uint64_t key_update[N_BUTTON];
uint64_t update_counter;

void check_buttons_states() {
    update_counter++;

	for(int i = 0; i < N_BUTTON; i++) {
        int pressed = !((*buttons_mem) & (1 << i));

        if (pressed == pressed_state[i]) continue;

        pressed_state[i] = pressed;
        key_update[i] = update_counter;
	}
}

bool pressed(int button) {
	return pressed_state[32 - __builtin_clz(button) - 1] && key_update[32 - __builtin_clz(button) - 1] == update_counter;
}

bool pressing(int button) {
	return pressed_state[32 - __builtin_clz(button) - 1];
}
\end{lstlisting}

A Figura \ref{demo-input} apresenta o teste implementado para checar o pressionamento dos botões do GBA. Para cada botão pressionado um \textit{pixel} vermelho aparece na tela. Nesta figura, os botões B, R, \textit{LEFT}, \textit{DOWN} e \textit{START} estão sendo pressionados simultaneamente. O código \ref{lst:inputtest} apresenta este teste.

\begin{figure}[H]
 \centering \includegraphics[keepaspectratio=true,scale=0.6]{figuras/demo-input.eps}
   \caption[Demonstração do pressionamento de botões no emulador]
    {Teste de pressionamento de botões no emulador.}
   \label{demo-input}
\end{figure}

\begin{lstlisting}[float,caption={Código de teste de \textit{input}.},label={lst:inputtest}]
#include "video.h"
#include "input.h"
#include "utils.h"

#define RED 0x0000FF

volatile unsigned short *vid_mem = (unsigned short *)0x6000000;

int main() {
	reset_dispcnt();
	set_video_mode(3);
	enable_background(2);

    while(1) {
        check_buttons_states();

        for(int i=0;i<=9;i++){
            int padding = i + 10;

            if (pressing(1 << i)) {
                for(int x=-2;x<=2;x++) {
                    for(int y=-2;y<=2;y++) {
                        vid_mem[(50 + x) * 240 + (20 + y + i * 10)] = RED;
                    }
                }
            }
            else {
                for(int x=-2;x<=2;x++) {
                    for(int y=-2;y<=2;y++) {
                        vid_mem[(50 + x) * 240 + (20 + y + i * 10)] = 0;
                    }
                }
            }
        }
    }

    return 0;
}
\end{lstlisting}

O código deste projeto se encontra no seguinte endereço: \url{https://bit.ly/2BQV6jf}.